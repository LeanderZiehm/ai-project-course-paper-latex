\documentclass{article}
\usepackage{graphicx}
\usepackage{hyperref}
\usepackage{amsmath}
\usepackage{enumitem}

\title{LLM Anonymization Techniques Recommender for Datasets (LLMANO-2)}
\author{Amin Akziz, Leander Ziehm, Azbabanu Engineer, Santiago Lema}
\date{July 2025}

\usepackage[a4paper, margin=1in]{geometry}

\usepackage{setspace}
\onehalfspacing

\begin{document}

\maketitle

\begin{center}
\begin{minipage}{0.85\textwidth}
\section*{Abstract}
The need for anonymizing datasets has grown significantly with the proliferation of machine learning and large language models (LLMs). However, many users lack a clear understanding of how to properly anonymize their data without significantly compromising its utility. In response to this challenge, we present LLMANO-2, a recommender system that leverages LLMs to guide users in anonymizing CSV datasets while preserving predictive performance. Our system provides tailored recommendations based on the specific characteristics of each dataset, aiming to strike a balance between privacy and usability. The motivation for this work stems from recent data privacy incidents that highlight the risks of improper anonymization and the growing demand for practical, intelligent tools to support privacy-aware data publishing.
\end{minipage}
\end{center}

\section{Introduction}

In today’s era of big data and artificial intelligence, vast amounts of personal and sensitive information are being collected and analyzed. This increase in data collection has made dataset anonymization a critical concern, as organizations must protect individual privacy while still deriving value from the data.

Notably, data anonymization is not just a best practice but often a legal requirement, privacy regulations like the EU’s General Data Protection Regulation (GDPR) and the California Consumer Privacy Act (CCPA) mandate that personal data be anonymized or otherwise protected if it’s to be used beyond its original purpose. Failure to properly anonymize data can lead to severe regulatory penalties as well as loss of user trust.

Recently, advances in Artificial Intelligence have introduced new possibilities for smarter anonymization approaches. In particular, Large Language Models (LLMs) offer a promising approach to tackle the privacy-utility trade-off. In alignment with these developments, our work proposes LLMANO-2, an LLM-powered anonymization technique based recommender for datasets. LLMANO-2 is designed to assist users in anonymizing tabular datasets by recommending appropriate techniques for each column of data. For a given dataset, the system leverages an LLM to analyze the column contents and identify which fields are likely sensitive and suggest how to transform or mask them. The recommendations aim to maximize privacy protection while minimizing the impact on the dataset’s utility for analysis or machine learning tasks. In the following sections we describe the architecture and methodology behind LLMANO-2, including how it processes datasets, generates anonymization recommendations, and evaluates the effectiveness of those recommendations.


\section{Background and Theory}

A dataset (for example, a CSV file or database table) typically consists of many records (rows) with multiple attributes (columns) describing each record. These attributes can include direct identifiers (such as names, email addresses, or national ID numbers) as well as indirect identifiers (such as dates of birth, gender, or ZIP codes) that might not identify someone on their own, but could do so when combined with other information.

Data anonymization refers to the process of modifying a dataset to remove or obscure personal identifiers so that individuals cannot be readily identified. It involves erasing or encrypting identifiers that connect an individual to stored data while retaining the data’s overall usefulness. The challenge is to ensure privacy without significantly degrading the utility of the data.

In practice, anonymization techniques may include removing or masking direct personal identifiers, generalizing values (for example, replacing an exact age with an age range), or pseudonymizing data by replacing real identifiers with artificial codes. These transformations allow organizations to use data for insights without exposing specific individuals.

\begin{figure}[h]
    \centering
    \includegraphics[width=0.65\linewidth]{images/figure1.png}
    \caption{Example of an anonymization technique applied to a simple dataset}
    \label{fig:anonymization-example}
\end{figure}

This is especially important when sharing datasets with third parties or releasing them publicly for research, as it enables data sharing and collaboration without compromising confidentiality.

Despite its importance, effective anonymization can be challenging. Research has shown that improperly anonymized datasets can still compromise privacy. Certain attributes are extremely sensitive, and even "anonymized" data can be re-identified. A famous statistic by Latanya Sweeney demonstrated that 87\% of the U.S. population could be uniquely identified using only three pieces of information, birth date, gender, and ZIP code, highlighting how a few quasi-identifiers can pinpoint individuals.

Another example where personal data was compromised is in the Netflix Prize dataset released in 2006, which contained movie ratings with personal identities removed, researchers were able to re-identify specific users by correlating the anonymized ratings with public information on IMDb (Internet Movie Database). This meant that personal viewing preferences (and even sensitive attributes inferred from them) of certain Netflix users became public, despite Netflix’s attempt to anonymize the data.

These cases show that simply removing obvious identifiers is often not sufficient. If enough data remains, determined adversaries can link supposedly anonymous records back to real individuals. Beyond the harm to individuals (such as exposing health information, financial status, or location history), such incidents expose organizations to legal liabilities and reputational damage. As data mining and cross-referencing techniques grow more sophisticated, re-identification attacks are becoming easier and more common. This puts pressure on data publishers to adopt stronger anonymization measures than were needed in the past.

At the same time, anonymization must be done carefully to preserve the usefulness of data. Over-aggressive anonymization, such as deleting or randomizing too many fields, can render a dataset nearly useless for analysis or model training, defeating the purpose of sharing the data. The more we strip away or distort to protect privacy, the less accurate or insightful the dataset becomes. A core challenge is finding the sweet spot where sensitive details are well-protected, yet the dataset still supports meaningful insights or predictions.

Achieving this balance often requires expertise and context, deciding which attributes to generalize or mask, and to what extent, without degrading the data’s integrity. Unfortunately, many users and organizations lack a clear understanding of how to achieve this. Choosing the right anonymization techniques depends on the data content, possible auxiliary data an attacker may use, and the intended use of the anonymized dataset. As a result, there is a growing need for tools and guidance to help data owners navigate this complexity.



\subsection{Identifiers and Quasi-Identifiers}

To give a broader context, we first define key terms related to data anonymization.
\textbf{Identifiers} are attributes in a dataset that can be used to uniquely identify an individual. They can be classified into two main categories:

\begin{itemize}
\item \textbf{Direct Identifiers} are attributes that can uniquely identify an individual without external information. Common examples include names, social security numbers, or email addresses.

\item \textbf{Indirect Identifiers}, also called \textit{quasi-identifiers}, are attributes that do not uniquely identify an individual on their own but can do so when combined with other quasi-identifiers. Examples include age, ZIP code, and gender.
\end{itemize}




\subsection{Anonymization Techniques}

Several anonymization techniques have been developed to reduce the risk of re-identification while maintaining the utility of data. These techniques vary in complexity and suitability depending on the nature of the dataset and the privacy requirements.

\begin{itemize}
\item \textbf{Suppression} involves removing data entirely from a dataset. For example, if a particular ZIP code is too rare and could identify someone, it might be deleted or replaced with a placeholder (e.g., “***”). This method is straightforward but can result in significant data loss if overused.

\item \textbf{Generalization} replaces specific values with broader categories. For instance, replacing an exact age of 29 with an age range such as “20–30,” or converting a full date of birth into just a year. Generalization helps reduce uniqueness in the data and is often used in combination with other techniques.

\item \textbf{Differential Privacy} takes a different approach by introducing mathematical noise to the data or the results of queries on the data. It provides strong theoretical guarantees that the inclusion or exclusion of a single individual’s data does not significantly affect the output, making it extremely difficult to infer information about any individual. Differential privacy is especially useful in scenarios involving aggregate data analysis or interactive queries.

\end{itemize}

These techniques are often used in combination and must be tailored to the specific context of the dataset. The choice of technique involves trade-offs between privacy protection and data utility, and selecting the appropriate method requires a clear understanding of the risks, the data's structure, and its intended use.


We also discuss user-driven anonymization processes and the limitations of manual or rule-based methods in ensuring privacy. In the remainder of this section, we outline the mathematical foundations underlying key privacy-preserving techniques in data publishing, including $k$-anonymity, $l$-diversity, and generalization strategies.

\subsection{$k$-Anonymity}

The concept of $k$-anonymity, introduced by Sweeney, is one of the foundational principles for privacy protection. A dataset is said to satisfy \textbf{$k$-anonymity} if each record is indistinguishable from at least $k-1$ other records with respect to the quasi-identifiers.

Formally, given a dataset $D$ with quasi-identifier attributes $Q_1, Q_2, ..., Q_m$, let $\pi$ be a projection onto the quasi-identifier space:

$$
\pi(D) = \{ (q_1, q_2, ..., q_m) \mid (q_1, ..., q_m) \in D \}
$$

Let $E$ be an equivalence class of records in $D$ that share the same quasi-identifier values:

$$
E = \{ r \in D \mid \forall i, \; r[Q_i] = v_i \}
$$

Then $D$ satisfies $k$-anonymity if:

$$
\forall E \subseteq D, \quad |E| \geq k
$$

This ensures that an adversary can not link any given record to fewer than $k$ individuals. However, $k$-anonymity alone does not prevent \textit{attribute disclosure}—when sensitive values within a group are too homogeneous.

\subsection{$l$-Diversity }

$l$-Diversity extends $k$-anonymity to protect against attribute disclosure by ensuring that each equivalence class has at least $l$ ``well-represented'' sensitive attribute values.

\paragraph{ Definition.}
An equivalence class $E$ is said to have \textit{distinct $l$-diversity} if it contains at least $l$ distinct values for the sensitive attribute $S$:

\[
|\{ s_i \mid s_i \in E \}| \geq l
\]

where $s_i$ denotes the sensitive value of record $i$ in equivalence class $E$.

\paragraph{Variants.}
\begin{itemize}
    \item \textbf{Distinct $l$-diversity:}
    \[
    |\{ s_i \}| \geq l
    \]
    Requires at least $l$ distinct sensitive values, regardless of their distribution or meaning.

    \item \textbf{Entropy $l$-diversity:}
    Defines diversity using entropy of the sensitive value distribution within $E$:
    \[
    H(E) = - \sum_{s \in S} p_s \log p_s \geq \log l
    \]
    where $p_s$ is the fraction of records in $E$ with sensitive value $s$. This ensures a balanced distribution.

    \item \textbf{Recursive (c, $l$)-diversity:}
    Limits the dominance of the most frequent sensitive value. Formally, for the sorted frequencies $f_1, f_2, ..., f_m$:
    \[
    f_1 < c \cdot (f_l + f_{l+1} + ... + f_m)
    \]
    where $c$ is a parameter controlling strictness. This prevents a single sensitive value from overwhelming others even if there are $l$ distinct values.
\end{itemize}

\paragraph{Why Needed.}
While $k$-anonymity prevents identity disclosure, it does not prevent attribute disclosure if all records in an equivalence class share the same sensitive value. $l$-Diversity mitigates this by ensuring diversity in sensitive attributes.

\paragraph{Limitations.}
\begin{enumerate}
    \item \textbf{Skewness Attack.} If the overall distribution is skewed (e.g. 99\% flu, 1\% cancer), $l$-diversity fails to prevent inference because the presence of rare values remains revealing.

    \item \textbf{Similarity Attack.} Even if an equivalence class has $l$ distinct sensitive values, if they are semantically similar (e.g. different types of cancer), meaningful privacy is not protected.
\end{enumerate}


\subsection{$t$-Closeness }

$t$-Closeness improves upon $l$-diversity by protecting against attribute disclosure through distributional similarity constraints.
\paragraph{ Definition.}
An equivalence class $E$ is said to have \textit{$t$-closeness} if the distance between the distribution of the sensitive attribute within $E$ ($P_E$) and the global distribution ($P_T$) is no more than a threshold $t$:

\[
D(P_E, P_T) \le t
\]

where:
\begin{itemize}
    \item $P_E$ is the probability distribution of sensitive attribute values in equivalence class $E$
    \item $P_T$ is the global distribution of the sensitive attribute in the entire dataset
    \item $D(\cdot)$ is a chosen distance measure, typically Earth Mover's Distance (EMD) or Kullback-Leibler Divergence.
\end{itemize}

\paragraph{Motivation.}
$l$-Diversity fails if equivalence class distributions are skewed compared to the overall data, enabling attribute inference attacks. $t$-Closeness mitigates this by bounding the distributional distance.

\paragraph{Common Distance Measures.}
\begin{itemize}
    \item \textbf{Earth Mover's Distance (EMD):} Measures the minimum amount of ``work'' required to transform one distribution into another, considering attribute semantics. It is formally defined as:
    \[
    EMD(P_E, P_T) = \inf_{\gamma \in \Gamma(P_E, P_T)} \int |x-y| d\gamma(x,y)
    \]
    where $\Gamma(P_E, P_T)$ is the set of all joint distributions with marginals $P_E$ and $P_T$.

    \item \textbf{Kullback-Leibler (KL) Divergence:}
    \[
    D_{KL}(P_E || P_T) = \sum_{i} p_E(i) \log \frac{p_E(i)}{p_T(i)}
    \]
    Note: KL-divergence is asymmetric and unbounded, limiting its practical use in $t$-closeness.

\end{itemize}

\paragraph{Example.}
If the global distribution is:
\[
P_T = \{ \text{Disease A}: 95\%, \text{Disease B}: 5\% \}
\]
but an equivalence class $E$ has:
\[
P_E = \{ \text{Disease B}: 100\% \}
\]
then even with $l$-diversity, an attacker learns an individual's disease with certainty. $t$-Closeness prevents this by ensuring $P_E$ remains close to $P_T$ within threshold $t$.

\paragraph{Advantages.}
\begin{itemize}
    \item Addresses both identity and attribute disclosure by enforcing global distribution similarity.
    \item Reduces risk of skewness and similarity attacks inherent to $l$-diversity.
\end{itemize}

\paragraph{Limitations.}
\begin{enumerate}
    \item \textbf{Utility Loss:} Strict $t$-closeness often requires heavy generalization or suppression, degrading data utility.
    \item \textbf{Threshold Selection:} Choosing an appropriate $t$ balancing privacy and utility is challenging.
    \item \textbf{Implementation Complexity:} Calculating EMD for categorical attributes requires defining semantic distances or hierarchies between categories.
    \item \textbf{Overprotection:} May restrict data utility disproportionately compared to marginal privacy gains in some contexts.
\end{enumerate}



\section{Architecture}
\label{sec:architecture}

\subsection{Design rationale}

LLMANO is meant to operate as an on-premise, language-centric assistant that guides non-experts through the anonymisation of tabular datasets.  Consequently the system must (i) accept queries in natural language, (ii) ingest a spreadsheet-like file, (iii) perform an automatic privacy screening before any user action, and (iv) expose a set of callable tools so the language model can compute metrics such as $k$-anonymity or $l$-diversity on demand.  These requirements favour a modular architecture: new capabilities should be added with minimal friction and without touching unrelated modules.  Expansion, for example adding a fresh privacy tool, must not require edits to the integration of the LLMs in the pipeline.  Open-source frameworks are preferred throughout; while the large language model itself may be commercial, the current build already includes an Ollama adapter that can point to a fully self-hosted model, yielding a pipeline that is open from end to end.

The platform is model-agnostic.  A thin abstract class, \texttt{BaseLLM}, exposes only the primitive operations needed by the rest of the application—token streaming, temperature control, and a low-level function call interface.  Concrete subclasses, here called adapters, encapsulate every provider-specific nuance, from authentication to tool-calling syntax.  Whenever a new model is required, the developer implements a single adapter; the surrounding code remains unchanged, and the user can switch models at run time.  Higher-level helpers such as \texttt{fuzzy\_match} are implemented \emph{outside} \texttt{BaseLLM} and rely exclusively on those primitive methods, so they work identically with every adapter.

\begin{figure}[h]
    \centering
    \includegraphics[width=0.85\linewidth]{images/class_diagram.pdf}
    \caption{Class diagram of \texttt{BaseLLM}, adapters, engine and orchestrator.}
    \label{fig:class-diagram}
\end{figure}

\subsection{Hardware notes}

Because the platform delegates inference to whichever model the user selects, resource consumption scales with parameter count.  All experiments so far have been conducted on Linux.  A local 8-billion-parameter Llama runs successfully on CPU, though latency is noticeable; GPU acceleration or a smaller checkpoint is advisable for interactive use.  Section~\ref{evaluation} reports measured runtimes.

\subsection{LLM-mediated decisions}

Balancing rich prompts against atomic decisions proved delicate.  Prior work shows that large language models perform well when deductively coding text against a fixed codebook~\cite{llm_content_analysis}.  We therefore implemented a helper, \texttt{fuzzy\_match}, which accepts (i) a natural-language context, (ii) a dictionary of codes, and (iii) the candidate attribute together with sample values.  The model must return the best-matching code.  This mechanism tags each column as a direct identifier (\texttt{DI}), quasi-identifier (\texttt{QI}), sensitive attribute (\texttt{SA}) or non-identifier (\texttt{NI}) and later chooses an appropriate anonymisation technique.  As observed by Chew \textit{et al.}~\cite{llm_content_analysis}, refining the code definitions with the same LLM improves downstream accuracy, a pattern repeated here.

\begin{table}[h]
    \centering
    \begin{tabular}{lp{0.75\linewidth}}
        \hline
        \textbf{Code} & \textbf{Definition}\\
        \hline
        DI & Direct identifiers that, by themselves, uniquely identify an individual.\\
        QI & Quasi-identifiers that enable re-identification only in combination with other data.\\
        SA & Sensitive attributes whose disclosure is harmful even without explicit identity.\\
        NI & Attributes that cannot reasonably re-identify an individual, even in combination.\\
        \hline
    \end{tabular}
    \caption{Example codebook (simplified) used by \texttt{fuzzy\_match}.}
    \label{tab:fuzzy-codebook}
\end{table}

Several approaches were tested to ensure the model returned well-formed data.  Embedding the target schema directly in the prompt produced unstable results; adding iterative validation loops improved reliability but quickly cluttered the adapters.  The final solution adopts LangChain’s structured-output feature: replies are validated against a \texttt{Pydantic} model such as
\begin{verbatim}
class FuzzyMatchResponse(BaseModel):
    code: str
    rationale: str | None
\end{verbatim}
This approach yields consistent parsing while keeping the adapters concise and remains fully open-source.

\subsection{Workflow}

The anonymization session begins with an intake form in which the user states the analysis goal, their existing knowledge of the dataset and any columns of particular concern.  The platform then inspects the file, extracts five representative rows plus summary statistics and forwards both dataset and form content to the model.  The first reply is a brief overview of the data; the second comprises one classification call per column, followed by anonymisation recommendations derived from an internal knowledge base.  Results are rendered as a colour-coded table—red for \texttt{DI}, yellow for \texttt{QI} and blue for \texttt{SA}—and accompanied by a concise markdown report.

Once the initial analysis is complete the system enters a conversational loop.  Each user query passes through the active adapter.  If the request references an implemented metric (for example \(k\)-anonymity), the adapter calls the matching Python routine and embeds the result in the reply.  After every turn the model proposes three plausible follow-up questions, using the full dialogue as context.

\begin{figure}[h]
    \centering
    \includegraphics[width=0.85\linewidth]{images/workflow_diagram.pdf}
    \caption{End-to-end workflow of LLMANO.}
    \label{fig:workflow-diagram}
\end{figure}

The output presented to the user consists of the chat transcript, the coloured table head and a recommendation pane listing a suggested anonymisation method for every non-\texttt{NI} column.

\subsection{Summary}

The architecture satisfies the original design goals: it runs on-premise, ingests tabular data, performs an automatic privacy audit and answers natural-language questions.  Modularity is maintained through an adapter layer that isolates provider details, while structured outputs guarantee that the application receives well-formed data.  The resulting system can be extended, audited and deployed without risking exposure of raw, non-anonymised records.


\section{Related Work}
\subsection{Application of LLM-Assisted Deductive Coding for Fuzzy Matching}

In our system, we implemented a \texttt{fuzzy\_match} function inspired by the deductive coding approach presented by Arnold et al. (2023) in \textit{LLM-Assisted Content Analysis}. Their methodology uses large language models (LLMs) to select a single best-fitting label from multiple predefined options, grounded in context and code definitions. We adapted this technique for robust, scalable option selection tasks in our pipeline.

\subsubsection{How We Used the Technique}

\paragraph{Problem Context}
Our task required selecting the most appropriate code or label for a given candidate input, from a predefined set of options, based on its meaning within a specific context. This is conceptually identical to deductive coding, where text data is categorized using an established codebook.

\paragraph{Inspiration from the Paper}
The paper demonstrated that LLMs perform effectively when prompted with:
\begin{itemize}
    \item An input text
    \item A list of options (codes) with clear definitions and examples
    \item Instructions to select the best fitting option only from the list
\end{itemize}
This structured prompt ensures the LLM leverages its semantic understanding to perform closed-set classification, avoiding invalid or out-of-scope outputs.

\paragraph{Our Implementation}
In our \texttt{fuzzy\_match} tool:
\begin{itemize}
    \item We provide the model with:
    \begin{itemize}
        \item A \textbf{context} explaining the selection task
        \item The \textbf{options dictionary}, mapping each code to its definition
        \item The \textbf{candidate input} requiring classification
    \end{itemize}
    \item We construct a structured prompt instructing the model to select the correct code strictly from the provided options.
    \item The LLM generates a JSON response containing:
    \begin{itemize}
        \item The selected \textbf{code}
        \item An optional \textbf{rationale} explaining its choice
    \end{itemize}
\end{itemize}

This mirrors the paper's method, where the LLM’s reasoning is grounded in the definitions and examples given for each code, allowing it to perform classification based on semantic alignment rather than purely lexical similarity.

\subsubsection{Key Principle Applied}


\textbf{Closed-set selection with definitions.} The technique enforces that the output is always within the valid set of codes, ensuring interpretability, consistency, and compatibility with downstream processing pipelines. This is critical for tasks such as qualitative coding, label assignment, or controlled classification where outputs outside the predefined set are invalid.

\subsection{Inspiration from JarviX}

Our LLMANO‑2 system draws inspiration from \textit{JarviX} , a no-code platform for tabular data analytics using LLMs. JarviX elegantly combines structured workflows, prompt-engineering logic, and AutoML integration to guide users through complex analytic tasks. We adopted several of its key design elements:

\begin{itemize}
    \item \textbf{LLM-driven, context-aware prompt flow.} As in JarviX, our system assesses the dataset schema and user goals to generate guiding questions and suggest relevant next actions dynamically .

    \item \textbf{Missing-context queries.} When crucial information is not provided, LLMANO‑2 automatically asks follow-up questions—mirroring JarviX’s Question Matcher module that detects ambiguity and clarifies requirements before analysis execution.
    \item \textbf{Future prompt suggestions.} Inspired by JarviX’s Analysis Consultant, which recommends subsequent lines of inquiry and visualizations, LLMANO‑2 likewise offers users a menu of potential follow-up prompts to deepen insight.
    \item \textbf{Hierarchical system architecture.} We adopted a modular architecture similar to JarviX’s pipeline decomposition—starting from data ingestion, to LLM-driven interpretation, to visualization and optimization—enabling flexible extension and improved maintainability.
\end{itemize}

\paragraph{Workflow (adapted from JarviX).}
...............


\paragraph{Prompt logic.}
Let $\mathcal{D}$ be the dataset schema and $u$ the user's initial prompt. JarviX first derives:
\[
\mathrm{ctx}_0 = \textit{Parser}(\mathcal{D}, u),
\]
then checks completeness:
\[
\textit{if } \mathrm{ctx}_0 \not\vdash \textit{sufficient} \text{ then ask }\Delta q,
\]
and iterates until $\mathrm{ctx}_k$ is sufficient. The final query is submitted to the LLM for execution. This controlled, looped clarification inspired our back-and-forth logic.

\paragraph{Summary.}
By adapting JarviX’s workflow, prompt refinement, and suggestion strategies, LLMANO‑2 gains structured dialog management and context-sensitive prompting—key to robust no-code analytics. We extended this foundation with advanced context tracking and conversational capabilities to enhance usability and reliability.
\subsection{Definitions and Notation Used in LLMANO (Based on SCORR)}

In developing LLMANO, we adopted the definitions, notation conventions, and assumptions from the paper \emph{Scoring System for Quantifying the Privacy in Re-Identification of Tabular Datasets (SCORR)} . Specifically, we use its classification of Direct Identifiers (DIs), Quasi-Identifiers (QIs), and Sensitive Attributes (SAs), as well as its formal dataset notation and risk analysis assumptions.

\paragraph{Notation.}
Let $n$ be the number of records, $m$ the number of QI attributes, and $p$ the number of distinct persons:
\[
\begin{array}{rl}
\text{Attributes} &= \{a_1, a_2, \ldots, a_{m+2}\} \\
\text{Person ID} &= a_1 \\
\text{QI} &= \{a_2, \ldots, a_{m+1}\} \\
\text{SA} &= a_{m+2}
\end{array}
\]
where $v_{ij}$ is the value of attribute $a_j$ in record $i$, with:
\[
\begin{array}{rl}
v_{i1} &= u_k \quad \text{(Person ID)} \\
QI_i &= \{v_{i2}, \ldots, v_{i,m+1}\} \\
SA_i &= v_{i,m+2}.
\end{array}
\]

\paragraph{Assumptions.}
LLMANO adopts SCORR’s assumptions that:
\begin{enumerate}
    \item The intruder knows the QI values of the target data subject.
    \item The intruder aims to infer the SA value associated with the target.
\end{enumerate}

\paragraph{Summary.}
These definitions, notation structures, and assumptions from \emph{Scoring System for Quantifying the Privacy in Re-Identification of Tabular Datasets} provide the foundation for LLMANO’s identifier recognition logic, privacy metric calculations, and risk analysis modules.

\section{Evaluation} \label{evaluation}
Our system allows users to:
\begin{enumerate}
    \item Upload CSV datasets
    \item Select columns as direct or quasi-identifiers (form includes definitions and examples)
    \item Receive LLM-generated anonymization recommendations (e.g., $k=10$)
    \item Optionally adjust granularity and generalization steps
\end{enumerate}
LLMs are prompted to recommend anonymization strategies based on dataset structure and user inputs. Screenshots and/or GitHub links will be included.

We test our system on multiple datasets:
\begin{itemize}
    \item Comparison against human-created anonymization
    \item Benchmarking against other anonymization algorithms
    \item Metrics: preservation of predictive accuracy, anonymity scores
\end{itemize}
Each configuration is tested 10 times to compute average performance. Evaluation is automated and requires no user interaction.

\section{Discussion}
\subsection{Limitations}
\label{sec:limitations}

The present prototype demonstrates the feasibility of an LLM–assisted anonymisation workflow, yet several constraints remain.  
First, the system relies on multiple, narrowly scoped prompts to achieve stable results.  Each additional prompt increases latency and, when a commercial model is used, raises monetary cost through higher token throughput.  A production deployment will therefore require a careful balance between prompt granularity and operational expense.  

Second, although the architecture permits arbitrary Python tools to be exposed to the language model, only a small set is currently implemented, limited to basic privacy metrics such as $k$-anonymity and $l$-diversity.  Expanding this library is essential for broader applicability.  

Third, the range of supported anonymisation techniques is still narrow.  At present the system can recommend generalisation, suppression and a differential-privacy layer, but it cannot yet prescribe more advanced transformations such as homomorphic encryption, local differential privacy or synthetic data generation.  A systematic comparison with these alternatives is left for future work.  

Finally, the current output is minimal.  Results appear only as a simple, chat-independent box that lists one suggested anonymisation method per attribute, and they are not yet downloadable in any report format.  The box is meant as a symbolic placeholder for a richer document that the LLM will ultimately compose, containing a narrative summary, detailed rationale and actionable next steps.  Implementing export functions and an editable report view remains future work.

\subsection{Future work}

Although LLMANO-2 already provides a flexible and modular setup for LLM-assisted anonymization, there is still plenty of room to grow and improve. Two areas stand out as especially promising for future development which are integrating open-source agentic frameworks, and moving away from heavy prompt engineering in favor of Retrieval-Augmented Generation (RAG) to support more scalable and efficient reasoning.

\paragraph{Agentic Orchestration and Structured LLM Tooling.}
As LLMs become more deeply embedded in structured applications, there is growing momentum behind the so called “agentic” systems. Setups where models do not just respond to prompts, but also take initiative, call tools, and manage multi-step workflows. Open-source frameworks like LangChain and CrewAI make this kind of functionality much easier to implement. For LLMANO-2, a natural next step would be to adopt these frameworks to handle tool-calling and decision-making more explicitly. Features like LangChain’s StructuredOutput and StateGraph could allow the system to coordinate complex tasks, such as risk analysis, column classification, and technique selection in a clean, modular way. This would move the workflow beyond a simple sequence of prompts into a flexible decision graph, making it easier to track, reuse, and extend each step of the process.

In addition, structured function-calling, supported by LangChain and similar agentic frameworks, lets us enforce strict output formats in a way that simple prompt validation often can not. This shifts the role of the LLM from just a reasoning engine to something closer to a programmable agent that can reliably carry out specific steps in the anonymization workflow. With tool-calling, the model could directly trigger external components like homomorphic encryption libraries or advanced generalization tools, without needing to encode all the logic in the prompt itself. This kind of structured setup fits well with modern hybrid AI practices, where symbolic routines and modular code work alongside probabilistic models to create more reliable and maintainable systems \cite{jarvix}.

\paragraph{From Prompting to Retrieval-Augmented Generation (RAG).}
Currently, LLMANO-2 depends heavily on carefully crafted prompts to guide its anonymization logic and provide context. While this approach has produced good results, it is also resource intensive, both in terms of token usage and the effort needed to maintain the prompts. A more scalable alternative is Retrieval-Augmented Generation (RAG), which separates the retrieval of relevant information from the actual generation process. Instead of packing all the logic into a single prompt, a RAG-based system can pull in relevant anonymization strategies, definitions, and examples from a vector store, allowing the model to work with just the most useful context for each task.

In a RAG-based setup, the choice of which anonymization method to use would be informed by embedded representations of past successful cases, legal requirements, and established best practices, all stored in a well curated knowledge base. This approach is especially useful when working across different domains like healthcare, finance, or education, where privacy expectations can vary widely. Tools like FAISS \cite{faiss2023} or Chroma can serve as the vector search engine behind this, enabling the system to retrieve relevant context without being limited by token constraints. Combined with LLM reasoning, this setup can improve both accuracy and reliability by grounding the model’s output in meaningful, domain-specific information \cite{llm_content_analysis}.

Adopting a RAG-based architecture would also make the system more modular. Domain-specific privacy guidelines could be stored in separate knowledge bases, allowing the model to retrieve only what’s relevant for each context, without needing to retrain or manually tweak prompt templates. This is particularly helpful when dealing with evolving regulations, like updates to GDPR interpretations, since changes can be made directly in the documents rather than in the application code.

\paragraph{Summary.}
Agentic orchestration and RAG complement each other well and offer solutions to two of LLMANO-2’s current limitations, its reliance on linear prompt flows and the brittleness of static prompt design. By combining tool-based reasoning with modular knowledge retrieval, the system could become more robust, easier to interpret, and better suited for scaling. These improvements would also support more interactive, human-in-the-loop workflows, where users can review, adjust, or fine-tune anonymization steps based on clear intermediate outputs.

Ultimately, future versions of LLMANO should aim to move beyond static prompt execution and become proactive, memory-enhanced agents capable of retrieving context on demand, planning anonymization strategies, calling tools as needed, and adapting flexibly to new domains and evolving privacy requirements.


\section{Conclusion}
\label{sec:conclusion}

This report presented LLMANO, a prototype that couples large language models with a lightweight, modular framework for column-level anonymisation guidance.  The system accepts a CSV dataset, applies an automatic screening based on a fixed codebook, and proposes one anonymisation strategy per attribute.  All components are deployable on-premise; integration of new language models or privacy tools requires only the implementation of an additional adapter.

The current implementation remains limited.  Only three anonymisation techniques—generalisation, suppression and a differential-privacy layer—are supported, and the tool library covers basic risk metrics. Addressing these constraints, extending the set of privacy routines and adding export functionality constitute clear next steps.

Even with these restrictions, the prototype validates the feasibility of an LLM-assisted workflow that keeps raw data local and separates provider detail from core logic.  The design can therefore serve as a starting point for more comprehensive, open-source solutions in future work.


\begin{thebibliography}{9}

\bibitem{li2018}
N.~Li \emph{et~al.},
“On sampling, anonymization, and differential privacy…,”
\emph{IEEE Trans.\ Knowledge and Data Engineering}, 2018.
\href{https://www.tandfonline.com/doi/full/10.1080/17579961.2018.1452176#abstract}{%
\texttt{doi:10.1080/17579961.2018.1452176}
}

\bibitem{esser2023}
E.~Esser \emph{et~al.},
“Privacy attacks on LLMs and how to defend,”
\emph{Science Advances}, 2023.
\href{https://www.science.org/doi/full/10.1126/sciadv.adn7053}{%
\texttt{doi:10.1126/sciadv.adn7053}
}

\bibitem{fake_news_dataset}
Clément Bisaillon,
\emph{Fake and Real News Dataset}. Kaggle.
\href{https://www.kaggle.com/datasets/clmentbisaillon/fake-and-real-news-dataset}{https://www.kaggle.com/datasets/clmentbisaillon/fake-and-real-news-dataset}

\bibitem{lending_club_dataset}
Wordsforthewise,
\emph{Lending Club Dataset}. Kaggle.
\href{https://www.kaggle.com/datasets/wordsforthewise/lending-club/data}{https://www.kaggle.com/datasets/wordsforthewise/lending-club/data}

\bibitem{student_demographics_dataset}
Anıl Gürbüz,
\emph{Student Demographics – Online Education Dataset}. Kaggle.
\href{https://www.kaggle.com/datasets/anlgrbz/student-demographics-online-education-dataoulad}{https://www.kaggle.com/datasets/anlgrbz/student-demographics-online-education-dataoulad}

\bibitem{hospital_discharges_dataset}
Bhautik Mangukiya,
\emph{Hospital Inpatient Discharges Dataset}. Kaggle.
\href{https://www.kaggle.com/datasets/bhautikmangukiya12/hospital-inpatient-discharges-dataset}{https://www.kaggle.com/datasets/bhautikmangukiya12/hospital-inpatient-discharges-dataset}

\bibitem{adult_income_dataset}
UCI Machine Learning Repository,
\emph{Adult Census Income Dataset}. Kaggle.
\href{https://www.kaggle.com/datasets/uciml/adult-census-income/data}{https://www.kaggle.com/datasets/uciml/adult-census-income/data}

\bibitem{titanic_dataset}
Yasser Hatab,
\emph{Titanic Dataset}. Kaggle.
\href{https://www.kaggle.com/datasets/yasserh/titanic-dataset}{https://www.kaggle.com/datasets/yasserh/titanic-dataset}

\bibitem{machanavajjhala2007}
Machanavajjhala, A., Kifer, D., Gehrke, J., and Venkitasubramaniam, M.,
\emph{$l$-diversity: Privacy beyond $k$-anonymity}. ACM Transactions on Knowledge Discovery from Data (TKDD), 1(1), 2007.
\href{https://doi.org/10.1145/1217299.1217302}{https://doi.org/10.1145/1217299.1217302}

\bibitem{li2007tcloseness}
Li, N., Li, T., and Venkatasubramanian, S.,
\emph{$t$-Closeness: Privacy beyond $k$-anonymity and $l$-diversity}. In Proceedings of the 23rd International Conference on Data Engineering (ICDE), 2007, pp. 106–115.
\href{https://doi.org/10.1109/ICDE.2007.367856}{https://doi.org/10.1109/ICDE.2007.367856}

\bibitem{fung2010survey}
Fung, B. C. M., Wang, K., Chen, R., and Yu, P. S.,
\emph{Privacy-preserving data publishing: A survey of recent developments}. ACM Computing Surveys (CSUR), 42(4), 2010, Article 14.
\href{https://doi.org/10.1145/1749603.1749605}{https://doi.org/10.1145/1749603.1749605}





\bibitem{llm_content_analysis}
Gong, M., Choudhury, S. R., Jiang, M., He, Y., Meng, Y., Li, X., Zhang, C., and Liu, H.,
\emph{LLM-Assisted Content Analysis: Demonstration and Evaluation}. arXiv preprint arXiv:2306.14924, 2023.
\href{https://arxiv.org/abs/2306.14924}{https://arxiv.org/abs/2306.14924}

\bibitem{scorr_paper}
Folz, J., Vidanalage, M. D., Aufschläger, R., Almaini, A., Heigl, M., Fiala, D., and Schramm, M.,
\emph{Scoring System for Quantifying the Privacy in Re-Identification of Tabular Datasets}. IEEE Access, 2025.
\href{https://ieeexplore.ieee.org/document/10973096}{https://ieeexplore.ieee.org/document/10973096}

\bibitem{jarvix}
Liu, S., Ji, Y., Tan, S., Zhang, J., Zhao, W., Liu, S., Chen, M., and Hu, Z.,
\emph{JarviX: An LLM No-code Platform for Tabular Data Analysis and Optimization}. arXiv preprint arXiv:2312.02213, 2023.
\href{https://arxiv.org/abs/2312.02213}{https://arxiv.org/abs/2312.02213}



\end{thebibliography}
\end{document}

